\documentclass[main]{subfiles}
\begin{document}
\chapter{実験全体を通して}
\section{実験概要}
この実験は測域センサを搭載したロボットの動作をプログラミングして、
要求される仕様(環境・条件)を満たす動作をロボットに行わせることを目的とする。
ロボットはYP-SpurによってPCと通信を行い、
プログラムからはライブラリを通じてアクセスする。
具体的なプログラミング・実行においては、
YP-Spur CoordinatorがインストールされたLinuxマシンを用いてC言語プログラムから
\textit{``ypspur.h''}に記述された関数を呼び出すことでこれを実現する。
以下に実行環境を示す。

\begin{table}[H]
	\centering
	\begin{tabular}{|c|c|}
		\hline
		PC & robozuki3 (ThinkPad X200)  \\\hline
		カーネル & Linux 4.4.0-71\\\hline
		OS & Ubuntu13.10\\\hline
		YP-Spur & 1.14.0\\\hline
	\end{tabular}
\end{table}

\section{学んだこと・考察}
ロボットのプログラムは現実の制約に縛られることが多く、
座標を取得しても回転の多い動作をさせると値が狂ってしまうことなどには随分悩まされた。
他にも、デバッグを行うにあたってSpurに覆い隠されている部分に入ってしまうと
途端難易度が急上昇した。

しかし、おかげで新たなライブラリを使用することには慣れたと思う。
最も重要なのは、まず仕様を確認してからプログラミングに取り組むことだ。
特にロボットはライブラリが複雑で、
現実世界を扱う必要があるので単位系も厳密に決まっている。
これを確認せずしてプログラムは書けない。

座標変換や最終課題での円の方程式の導出などで
線形代数で習得した知識を生かすことができた。
単純なコードではあるが、知識をうまく活用しセンサと組み合わせて物体の中心推定まで行った。

C言語のプログラムを書く力も上達した。
隋分長いコードになってしまったのでライブラリ化することも考えたが、
最終課題に必要な機能の実装が間に合わなかったのでかなわなかった。
最終課題の重量は突然これまでと比較できないほど重い実装を行い非常に辛かったが、
試行錯誤しながら自ら学び、またTAに教えてもらいながら
自分にできるだけの力を出し切って挑戦できた。

\section{謝辞}
TAの\TAa さんからは課題に行き詰まったときに多くの助言をいただき
設計に関する深い部分に関しても親身に評価・改善してくださりました。
\TAb さんにはデバッグに協力していただいただけでなく、
授業時間外にも実験について相談に乗っていただきました。
ここに感謝の意を表します。
\end{document}
