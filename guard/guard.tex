\documentclass[main]{subfiles}
\begin{document}
\chapter{ポールを周回しながら人が近づくと威嚇する}

これは最終課題5「警備員」の課題です。

\section{課題概要}
ロボットはポールを中心とした半径50cmの円を反時計回りに周回する。
ポールの直径は約12cm。
初期位置の制約は定められていないので、
前方もしくは左方にポールを見るような地点ならば
任意の場所からスタート可能なように設計を行った。

周回の最中に人がセンサに反応した場合、人とポールの間に立ち威嚇を行う。
威嚇の方法については定められていなかったので、
ポールから2m以内の範囲で追いかけるような設計にした。

人を見失ったり、ポールから2m以上離れてしまった場合は
再びポールの周回に戻る。

これを繰り返すことになる。

\section{解法}
\section{結果}
\section{考察}

\end{document}
