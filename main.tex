\documentclass[lualatex, a4paper, base=11pt, jafont=auto, ja=standard]{bxjsreport}

\usepackage{subfiles}
\usepackage{url}
\usepackage{float}
\usepackage{graphicx}
\usepackage{tikz}
\usepackage{hyperref}
\usepackage{newtxtext, newtxmath}
\usepackage{listings}
\usepackage{comment}
\usepackage[B]{thesis}	% 外観調整用
\usepackage[B]{cover}	% 表紙用

% listings設定
\lstset{
  basicstyle=\footnotesize\tt,
  breakatwhitespace=false,
  captionpos=b,
  extendedchars=true,
  frame=single,
  numbers=left,
  language=C,
  keywordstyle=\bf,
  showspaces=false,
  showstringspaces=false,
  showtabs=false,
  tabsize=4
}


\newcommand{\myname}{橘・シルフィンフォード}
\newcommand{\TAb}{日蓮}
\newcommand{\TAa}{空海}
	% 公開用

%% タイトル
\title{移動ロボットの行動プログラミング}
\papertype{ソフトウェアサイエンス実験B}
\institution{情報科学類}
\grade{3}
\studentno{201500000}
\author{\myname }
\era{平成}
\erayear{29}
\yearandmonth{2018年2月}

\begin{document}
\maketitle
\pagestyle{empty}
\tableofcontents
\newpage
\pagestyle{plain}
\setcounter{page}{1}

\subfile{intro/intro.tex}
\subfile{basic/basic.tex}
\subfile{wall/wall.tex}
\subfile{urg/urg.tex}
\subfile{guard/guard.tex}

\pagestyle{empty}
\chapter{添付資料}
\section{8の字走行}
\lstinputlisting[caption=8の字走行]{src/eight.c}
\newpage
\section{前方1m以内に障害物を見つけたら停止}
\lstinputlisting[caption=前方1m以内に障害物を見つけたら停止]{src/stop1m.c}
\newpage
\section{左側にある壁に沿って走らせる兼柱の周りを周回する}
\lstinputlisting[caption=左側にある壁に沿って走らせる]{src/wall50cm.c}
\newpage
\section{座標変換したセンサデータをプロットする}
\lstinputlisting[caption=座標変換したセンサデータをプロットする]{src/urgprint.c}
\newpage
\section{ポールを周回しながら人が近づくと威嚇する}
\lstinputlisting[caption=最終課題5「警備員」]{src/guardian.c}

\end{document}
